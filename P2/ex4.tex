\documentclass{article}

\usepackage[small,compact]{titlesec}
\usepackage[backend=biber]{biblatex}
\usepackage[spanish]{babel}
\usepackage{epsfig}
\usepackage{array}
\usepackage{xfrac}
\usepackage{amsthm}
\usepackage{amsmath}
\usepackage{amssymb}
\usepackage{todonotes}
\usepackage{centernot}
\usepackage{textcomp}
\usepackage{blindtext}
\usepackage{centernot}
\usepackage{wasysym}
\usepackage{siunitx}
\usepackage{minted}
\usepackage[letterpaper]{geometry}
%\usepackage{multicol}
\usepackage{color}
%\usepackage[table]{xcolor}
\usepackage{amsfonts}
\usepackage{mathtools}
\usepackage{multirow}
\usepackage[small,it]{caption}
\usepackage{titling}
\usepackage{graphicx}
\usepackage[short]{optidef}
%\bibliographystyle{plain}
%\bibliographystyle{babplain}
\usepackage{filecontents}
\usepackage{titlesec}
\usepackage[section]{placeins}
\usepackage[hidelinks]{hyperref}
\usepackage{fancyhdr}
\usepackage{cancel}
\usepackage{abstract}
\usepackage{minted}

\sisetup{output-exponent-marker=\textsc{e}}

\captionsetup[table]{name=Table}


%\usepackage[makestderr=true]{pythontex}
%\restartpythontexsession{\thesection}

%\setpythontexoutputdir{./Temp}


\addbibresource{Bibliography.bib}

\pagestyle{fancy}
\usepackage[utf8]{inputenc}
\fancyhf{}
\fancyhead[c]{\textbf{\@title}}
\fancyfoot[c]{\thepage}
\def\Section {\S}

\newcommand\tstrut{\rule{0pt}{2.4ex}}
\newcommand\bstrut{\rule[-1.0ex]{0pt}{0pt}}

\DeclareMathOperator*{\argmax}{arg\,max}
\DeclareMathOperator*{\argmin}{arg\,min}

\setlength{\droptitle}{-4em}
\newcommand{\squishlist}{
 \begin{list}{$\bullet$}
  { \setlength{\itemsep}{0pt}
     \setlength{\parsep}{3pt}
     \setlength{\topsep}{3pt}
     \setlength{\partopsep}{0pt}
     \setlength{\leftmargin}{1.5em}
     \setlength{\labelwidth}{1em}
     \setlength{\labelsep}{0.5em} } }


\newcommand{\squishlisttwo}{
 \begin{list}{$\bullet$}
  { \setlength{\itemsep}{0pt}
    \setlength{\parsep}{0pt}
    \setlength{\topsep}{0pt}
    \setlength{\partopsep}{0pt}
    \setlength{\leftmargin}{2em}
    \setlength{\labelwidth}{1.5em}
    \setlength{\labelsep}{0.5em} } }

\newcommand{\squishend}{
  \end{list}  }
\footskip = 50pt
\setlength{\skip\footins}{10pt}

\newcounter{proofc}
\renewcommand\theproofc{(\arabic{proofc})}
\DeclareRobustCommand\stepproofc{\refstepcounter{proofc}\theproofc}
\newenvironment{twoproof}{\tabular{@{\stepproofc}c|l}}{\endtabular}

\usemintedstyle{tango}
 %% The usual stuff that sits
 %% between \documentclass
 %%    and \begin{document}

%\hypersetup{
%    bookmarks= \quadtrue,         % show bookmarks bar?
%    unicode= \quadfalse,          % non-Latin characters in Acrobat’s bookmarks
%    pdftoolbar= \quadtrue,        % show Acrobat’s toolbar?
%    pdfmenubar= \quadtrue,        % show Acrobat’s menu?
%    pdffitwindow= \quadfalse,     % window fit to page when opened
%    pdfstartview= \quad{FitH},    % fits the width of the page to the window
%    pdftitle= \quad{My title},    % title
%    pdfauthor= \quad{Author},     % author
%    pdfsubject= \quad{Subject},   % subject of the document
%    pdfcreator= \quad{Creator},   % creator of the document
%    pdfproducer= \quad{Producer}, % producer of the document
%    pdfkeywords= \quad{keyword1} {key2} {key3}, % list of keywords
%    pdfnewwindow= \quadtrue,      % links in new window
%    colorlinks= \quadfalse,       % false: boxed links; true: colored links
%    linkcolor= \quadred,          % color of internal links (change box color with linkbordercolor)
%    citecolor= \quadgreen,        % color of links to bibliography
%    filecolor= \quadmagenta,      % color of file links
%    urlcolor= \quadcyan           % color of external links
%}

%\addbibresource{References.bib}


\begin{document}
 %\thispagestyle{plain}
 \def\maketitle{%\twocolumn[%
 \thispagestyle{plain}
 \vspace{-10ex}
 \hrule
 \bigskip
 \begin{center}
 {\Large{\textbf{\@title}}}
 \end{center}
 \bigskip
 \hrule

 \bigskip

 \begin{flushleft}
 \textbf{\normalsize{Edgar Andr\'{e}s Margffoy Tuay}} \hfill 201412566
 \\
 \vspace{5pt}
 Universidad de los Andes \hfill Concurrency, Parallelism and Distribution
 \\
 \vspace{5pt}
\hfill \today \\ 
 \end{flushleft}
 %\hspace{265.2pt}
 %\bigskip
 %\bigskip
 }
\def\title#1{\def\@title{#1}}
\title{\textit{Exam 2}}



% \squishlist    %% \begin{itemize}
%\item First item
%%\item Second item
%%\squishend     %% \end{itemize}
 %% The rest of the paper (with no maketitle)
\maketitle

\section{MergeSort: Scala and Elixir implementation comparison}
%\begin{listing}
\begin{listing}
	\begin{minted}{scala}
	def sort(from: Int, until: Int, depth: Int): Unit = {
		if (depth == maxDepth) {
			quickSort(xs, from, until - from)
		} else {
			val mid = (from + until) / 2
			val right = task {
				sort(mid, until, depth + 1)
			}
			sort(from, mid, depth + 1)
			right.join()
	
			val flip = (maxDepth - depth) % 2 == 0
			val src = if (flip) ys else xs
			val dst = if (flip) xs else ys
			merge(src, dst, from, mid, until)
		}
	}
	\end{minted}
	\caption{Fork-Join implementation of Mergesort on Scala}
	\label{lst:L1}
\end{listing}

\begin{listing}
	\begin{minted}{elixir}
	def sort(list) do 
	    if length(list) <= cutoff do
	        Enum.sort(list)
	    else
	        mid = div length(list), 2 
	        {left, right} = Enum.split(list, mid)  
	        r_pid = Task.async(fn -> sort(right) end)
	        left = sort(left)
	        right = Task.await(r_pid)
	        merge(left, right)
	    end 
	end
	\end{minted}
	\caption{Fork-Join implementation of Mergesort on Elixir}
	\label{lst:L2}
\end{listing}

\begin{listing}
	\begin{minted}{elixir}
    def sort(list) do 
        mid = div length(list), 2 
        {left, risght} = Enum.split(list, mid) 
        l_pid = Task.async(fn -> sort(left) end) 
        r_pid = Task.async(fn -> sort(right) end) 
        merge(Task.await(l_pid), Task.await(r_pid)) 
    end
	\end{minted}
	\caption{Naïve parallel implementation of Mergesort on Elixir}
	\label{lst:L3}
\end{listing}

With respect to Listing \ref{lst:L2}, the Scala implementation of mergesort (Listing \ref{lst:L1}) is equivalent, as it uses Fork-Join to paralelize the splitting and merging of each sublist (Left and Right), while it sorts a list sequentially when its length is lesser or equal than the cutoff preset value. In contrast, the naïve implementation of mergesort on Elixir, as shown on Listing \ref{lst:L3} spawns a process for each split of a list, until each split has a single element. In all three implementations, the merge pass is done over the same process that forked the process for sorting the left sublist.

\subsection{Parallel Complexity}
If the merge pass is not parallelized for both Elixir implementations, then the work complexity would be $\mathcal{O}(n)$, whereas the span complexity would correspond to $\mathcal{O}(log n)$. In contrast, the parallel complexity using binary search-based merge would account for an overall complexity of:

\squishlist
\item Span: $\mathcal{O}(\log^3(n))$
\item Work: $\mathcal{O}(n \log(n))$
\squishend   

With respect to the Scala implementation, as it does the merge pass on a sequential fashion, its work complexity corresponds to $\mathcal{O}(n)$, while its span corresponds to $\mathcal{O}(log(n))$. 

\subsection{Could it be implemented on Elixir}
As all lists and variables (Including parallel collections) are immutable on Elixir, the Scala implementation cannot be ported as-is, due to its reliance on memory manipulations (In-place modifications and memory copying) for optimal running time. Thus, there would be no time execution reduction if it was to be implemented on Elixir, \textit{i.e.,} Listing \ref{lst:L2}. 
%\begin{alignat}{2}
%&\tilde{H} = \begin{bmatrix}
%\mathbf{A} & \mathbf{t} \\
%\mathbf{u}^{T} & \alpha
%\end{bmatrix} & \quad   \label{eq:e1}
%\end{alignat}



%\begin{figure}[!htbp]
%\centering
%\includegraphics[scale=0.3]{./Assets/1.png}
%\caption{Traza de Wireshark que presenta la emisión y recepción de paquetes ICMP enviados a un conjunto de %clientes presentes en la misma red.}
%\end{figure}


%\section{Diseño de filtros ideales}

%\begin{alignat}{2}
%h &= \begin{bmatrix}
%1 & 1 & 1 \\
%1 & 1 & 1 \\
%1 & 1 & 1 
%\end{bmatrix} \label{eq:e6}
%\end{alignat}


% \begin{equation}
% \begin{aligned}
% f ~:~ &\mathbb{R} &\longrightarrow ~ &\mathbb{R} \label{eq:e6} \\
%     &t &\longmapsto ~ &f(t)
% \end{aligned}
% \end{equation}
% \begin{equation}
% \begin{aligned}
% x ~:~ &\mathbb{Z} &\longrightarrow ~ &\mathbb{R} \label{eq:e7} \\
%     &n &\longmapsto ~ &x[n]
% \end{aligned}
% \end{equation}



% \begin{figure*}[!htbp]
% \centering
% \epsfig{file=./Assets/Discrete.pdf,width=1.0\linewidth,clip=}
% \caption{Ejemplos de señales discretas}
% \label{Fig:F3}
% \end{figure*}




%\bibliography{biblios} \nocite{*}
%\newpage
%\nocite{*}
%printbibliography



%\newcounter{proofc}
%\renewcommand\theproofc{(\arabic{proofc})}
%\DeclareRobustCommand\stepproofc{\refstepcounter{proofc}\theproofc}
%\newenvironment{twoproof}{\tabular{@{\stepproofc}c|l}}{\endtabular}


\end{document}